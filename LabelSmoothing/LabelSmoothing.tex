\documentclass{article}
\usepackage[UTF8]{ctex}
\usepackage{geometry}
\geometry{a4paper,left=1cm,right=1cm,top=1.5cm,bottom=2cm}
\usepackage{amsmath}
\usepackage{booktabs}
\usepackage{amsfonts}
\usepackage{graphicx}
\usepackage{float}
\usepackage[utf8]{inputenc}
\usepackage[T1]{fontenc}
\usepackage{tcolorbox}
\begin{document}
\title{\textbf{LabelSmoothing原理}}
\maketitle
\centerline{\author{Huacheng Li}}

\section{背景介绍: OneHot 编码}
神经网络的输出成为Logits,间记为$z$, 经过Softmax后转化为和为1的形式,记为$\hat{y}$,真实的target记为$y$, $K$为总的分类的类别的数量。

\begin{tcolorbox}[title=Logits]
    logits是指一件事情发生与不发生的比值的对数。假定时间发生的概率为$p$,那么该事件的logits为
    $$\text{logits}(p) = \log \frac{p}{1-p}$$
    在深度学习中,softmax会对输入进行归一化处理,则第$i$个类的概率
    $$p(i) = \frac{\exp (a_i)}{\sum_{j\in K} \exp (a_j)}$$
    $\{a_0, a_1, \ldots, a_K\}$就是logits。因此logits可以理解为为归一化的概率
\end{tcolorbox}

当损失函数为交叉熵,且target的编码和为1时,导数为$\hat{y}_i-y_i$。

\begin{tcolorbox}[title=求导过程]
    根据softmax公式和交叉熵公式,$p_i$为预测概率,$q_i$是真实概率
    $$p_i = \text{softmax}(z_i) = \frac{\exp (z_i)}{\sum_{j \in K} \exp (z_j)}$$
    $$Loss = - \sum_{i\in K} = q_i \log (p_i)$$

    求导:
    1. 当分类$i==k$时:
    \begin{equation*}
        \begin{split}
            \frac{\partial p_i}{\partial z_k} &= \frac{\partial p_i}{\partial z_i} = \frac{\partial \frac{\exp(z_i)}{\sum_{j \in K} \exp (z_j)}}{\partial z_i} = \frac{\exp(z_i)\sum_{j \in K}\exp (z_j) - \exp(z_i)\frac{\partial \sum_{j \in K} \exp(z_j)}{\partial z_i}}{[\sum_{j \in K} \exp (z_j)]^2} \\
            &= \frac{\exp (z_i)}{\sum_{j \in K} \exp(z_j)} - \frac{[\exp (z_i)]^2}{[\sum_{j \in K} \exp (z_j)]^2} = p_i -p_i^2
        \end{split}
    \end{equation*}

    2. 当$i \neq k$时:
    \begin{equation*}
        \begin{split}
            \frac{\partial p_i}{\partial z_k} &= \frac{\partial \frac{\exp (z_i)}{\sum_{j \in K} \exp (z_j)}}{\partial z_k} = \frac{0 \times \sum_{j\in K} \exp(z_j) - \exp(z_i) \frac{\partial \sum_{j \in K} \exp(j)}{\partial z_k}}{[\sum_{j \in K} \exp (z_j)]^2} \\
            &= - \frac{\exp(z_i) \exp (z_k)}{[\sum_{j \in K} \exp(z_j)]^2} = -p_i \times p_k
        \end{split}
    \end{equation*}

    3. 交叉熵求导
    \begin{equation*}
        \begin{split}
            \frac{\partial Loss}{\partial p_i} &= -q_i \partial \sum_{j\in K} \log (p_i) = -q_i \partial \log (p_i) = - q_i \times \frac{1}{p_i} = - \frac{q_i}{p_i}
        \end{split}
    \end{equation*}

    4. 对logits求导
    \begin{equation*}
        \begin{split}
            \frac{\partial Loss}{\partial z_i} &= \frac{\partial Loss}{\partial p_i} \times \frac{\partial p_i}{\partial z_i} + \sum_{j \neq i}(\frac{\partial Loss}{\partial p_j} \times \frac{\partial p_j}{\partial z_i}) = - \frac{q_i}{p_i} \times (p_i - p_i^2) + \sum_{j \neq i} (\frac{q_j}{p_j} \times p_j \times p_i) \\
            &= -q_i \times(1-p_i) + \sum_{j \neq i}(q_j \times p_i) = \sum_{j \in K} = -q_i + q_i \times p_i + \sum_{j \neq i}(q_j \times p_i) \\
            &= \sum_{j \in K} (q_j \times p_i) - q_i = p_i \sum_{j \in K}(q_j) - q_i = p_i-q_i
        \end{split}
    \end{equation*}
\end{tcolorbox}

假定总共有$K$个类,则可以如下表示:
\begin{equation}
    \begin{cases}
        \hat{y}_i = \frac{\exp (z_i)}{\sum_{j\in K} \exp(z_j)} \\
        \frac{\partial l}{\partial z_i} = \hat{y}_i - y_i
    \end{cases}
    \label{EQ:EQ1}
\end{equation}
根据真实情况,我们应当令正确类$\hat{y}_{true}=1$, 错误类$\hat{y}_{false}=0$,因此可以写为以下两式:
\begin{equation}
    \begin{cases}
        \hat{y}_{true} = \frac{\exp (z_{true})}{\sum_{j \in K} \exp (z_{j})} = 1\\
        \hat{y}_{false} = \frac{\exp (z_{false})}{\sum_{j \in K} \exp (z_{j})} = 0\\
    \end{cases}
    \label{EQ:EQ3}
\end{equation}
根据公式 (\ref{EQ:EQ3}) 可以得到下式:
\begin{equation}
    \begin{split}
        \exp(z_{true}) = \exp (z_{true}) + \sum_{j \neq true} \exp(z_j) \\
        \rightarrow \sum_{j \neq true} \exp(z_j) = 0 \\
        \rightarrow \exp (z_j)_{j \neq true} = 0 \\
        \rightarrow z_{false} = - \infty
    \end{split}
\end{equation}

\begin{tcolorbox}[title=Conclusion]
    因此在target为one-hot编码, 损失函数为交叉熵的情况下,$z_{true} \rightarrow C, z_{false} \rightarrow - \infty$。
    这表示错误类的logits为负无穷,正确类的logits为常数。这种最有情况一般是不能达到的, 且$z_{true} \gg z_{false}$。
    根据\cite{DBLP:conf/cvpr/SzegedyVISW16}观点,这种情况下会出现两个非常不好的性质:
    \begin{itemize}
        \item 导致过拟合,将所有概率都赋值给了真值,泛化能力下降
        \item 要求真值对应的logits要远远大于其他值的logits, 但导数 $\frac{\partial l}{\partial z_i}$是有界的,也就是数值不会很大。这意味着要更新很多次
    \end{itemize}
\end{tcolorbox}

\section{LabelSmoothing}
LabelSmoothing是\cite{DBLP:conf/cvpr/SzegedyVISW16}提出的。作者应该是认为:蒸馏改变了学习的真值,为了能够获得更好的结果,但是需要准确率更高的教师网络;如果现在想要训练出一个准确率最高的模型,要是没有网络能给我知识,所以就通过LabelSmoothing学习一种简单的知识。

LabelSmoothing 的编码形式如下式所示,其中$\epsilon$是超参数,一般取值为0.1
\begin{equation}
    y_i = \begin{cases}
        1-\epsilon  & if \quad i==true \\
        \frac{\epsilon}{K-1} & otherwise\\
    \end{cases}
    \label{EQ:EQ4}
\end{equation}
对公式(\ref{EQ:EQ4})求导,类似公式(\ref{EQ:EQ3})我们可以得到下式
\begin{equation}
    \begin{cases}
        \frac{\exp(z_{ture})}{\exp(z_{true} + \sum_{j \neq true} \exp(z_j))} = 1 - \epsilon \\
        \frac{\exp (z_{false})}{\sum_{j \in K} \exp (z_j)} = \frac{\epsilon}{K-1}\\
    \end{cases}
\end{equation}
因为正确的类只有1个,错误的类有$K-1$个, 且在解析解的情况下,错误类的概率近乎相等。因此可以得到下式:
\begin{equation}
    \begin{split}
        \exp(z_{true}) = (1-\epsilon) \exp(z_{true}) + (1-\epsilon)(K-1)\exp(z_{false}) \\
        \rightarrow \epsilon \exp(z_{true}) = (1-\epsilon)(K-1)\exp(z_{false})\\
        \rightarrow z_{true} = \log (\frac{(K-1)(1-\epsilon)}{\epsilon}) + z_{false}\\
    \end{split}
\end{equation}
可以令$z_{false}$为$\alpha$,那么在导数等于0的情况下,logits的取值为:
\begin{equation}
    z_i^{*} = \begin{cases}
        \log (\frac{(K-1)(1-\epsilon)}{\epsilon}) + \alpha & if \quad i=y \\
        \alpha & otherwise\\
    \end{cases}
\end{equation}

\begin{tcolorbox}[title = Conclusion]
    One-Hot编码需要错误类的logits趋向于负无穷,这样会导致正确类和错误类的输出误差很大,网络泛化能力不强。
    并且因为网络训练时一些正则化的存在,logits的输出很难是负无穷的。
    LabelSmoothing编码方式只要正确类和错误类有一定的数值误差即可。
\end{tcolorbox}

\bibliographystyle{apalike}
\bibliography{ref}

\end{document}